\documentclass[12pt]{article}
\usepackage[paperheight=11.5in,paperwidth=8.5in]{geometry}
\usepackage{amsmath}
\usepackage{amssymb}
\usepackage{version}
\usepackage{amsthm}
\usepackage{listings}
\usepackage{amscd}
\usepackage{amsfonts}
\usepackage{graphicx}%
\usepackage{fancyhdr}
\usepackage{pgfplots}
\includeversion{question}\includeversion{solution}


\newtheorem{theorem}{Theorem}[section]
\newtheorem{corollary}[theorem]{Corollary}
\newtheorem{conjecture}{Conjecture}
\newtheorem{lemma}[theorem]{Lemma}
\newtheorem{proposition}[theorem]{Proposition}
\theoremstyle{definition}
\newtheorem{definition}[theorem]{Definition}
\newtheorem{finalremark}[theorem]{Final Remark}
\newtheorem{remark}[theorem]{Remark}
\newtheorem{example}[theorem]{Example}

\textwidth6.3in

\setlength{\topmargin}{0in} \addtolength{\topmargin}{-\headheight}
\addtolength{\topmargin}{-\headsep}

\setlength{\oddsidemargin}{0in}

\oddsidemargin  0.0in \evensidemargin 0.0in

\pagestyle{fancy}\lhead{CS598} \rhead{13 November 2013}
\chead{{\large{\bf Dhashrath Raguraman} (raguram2)}} 


\begin{document}
\centerline {{\bf CS598 AITR Assignment 5 }}
\begin{solution}
 {\bf MapReduce programming and word association discovery } \\
\begin{itemize}

\item Code written and tested on Single Node Hadoop Cluster set up on OSX
\item API Version: 1.2.1  (this is different from the altocumulus box which has 0.19)
\item Java Files : wordCount.java , pairCount.java , mutualinformation.java

\item Data Set	: Running on a 4000 line pruned data set of apsrc.txt

\item Expected Outputs: wc.dat (wordcount) ,wpc.dat(wordpaircount) and mutualinfo.txt(file containing result)

\item {\bf Where can you find the outputs}: (hadoop relative) /home/raguram2/
\begin{verbatim}
hadoop fs -ls /home/raguram2
\end{verbatim}

{\bf Where can you find the code}: (linux default home path) /home/raguram2/
\end{itemize}

{ \bf Steps in running code}
\begin{itemize}
\item {\bf Step 1}
\newline Generate Pruned data set using head -4000  apsrc.txt 
\item {\bf Step 2}
\newline Generate jar using hadoop libraries  
\begin{verbatim}
javac -classpath /usr/local/Cellar/hadoop/1.2.1/libexec/lib/commons-cli-1.2.jar:
/usr/local/Cellar/hadoop/1.2.1/libexec/hadoop-core-1.2.1.jar  -d 
miclasses *.java && jar -cvf mi.jar -C miclasses/ .
\end{verbatim}
\item {\bf Step 3}
\newline Use the wordCount class that uses map reduce and derive counts on occurence in total documents and total number of occurrences.Word Counts go into "wc/" folder .You can find the wc.dat file used in my hadoop relative path /home/raguram2/wc.dat
\begin{verbatim}
hadoop jar mi.jar wordCount apsrc_4k.txt wc
\end{verbatim}

\item {\bf Step 4}
\newline Use the pairCount class that uses map reduce and derive pair counts. Pair Counts go into wpc/ folder .You can find the wpc.dat file used in my hadoop relative path /home/raguram2/wpc.dat
\begin{verbatim}
hadoop jar mi.jar pairCount apsrc_4k.txt wpc
\end{verbatim}
\item {\bf Step 5}
\newline Use the mutualinformation class and derive mutual information for a given word pair as per assignment instructions. Mutual Information gets written in mutualinfo.txt
\begin{verbatim}
hadoop jar mi.jar mutualinformation apsrc_4k.txt wc.dat wpc.dat mutualinfo.txt
//note that wc.dat and wpc.dat are renamed results inside the wc/ and wpc/ results
\end{verbatim}

I have attached the source files as well as sample output files (100 lines) of wc.dat,wpc.dat and mutualinfo.txt



\end{itemize}

\end{solution}
\end{document}
